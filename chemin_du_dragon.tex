%====================================================================================================
\documentclass[a4paper,12pt]{article}
%====================================================================================================

% Packages
\usepackage{amsmath}
\usepackage{amssymb}
\usepackage[francais]{babel}
\usepackage{multicol}
\usepackage[margin=2.0cm]{geometry}
\usepackage[nameinlink,capitalise,noabbrev]{cleveref}

% Quelques abreviations utiles
\def \be {\begin{equation}}
\def \ee {\end{equation}}
\def \bs {\begin{equation} \left\{  \begin{aligned}}
\def \es {\end{aligned} \right. \end{equation}}

\newcommand{\set}[1]{\left\{#1\right\}}
\newcommand{\minuss}[1]{\setminus\left\{#1\right\}}

\newcommand{\minusioo}[1]{\setminus\left\rbrack#1\right\lbrack}
\newcommand{\minusiof}[1]{\setminus\left\rbrack#1\right\rbrack}
\newcommand{\minusifo}[1]{\setminus\left\lbrack#1\right\lbrack}
\newcommand{\minusiff}[1]{\setminus\left\lbrack#1\right\rbrack}

\newcommand{\intoo}[1]{\left\rbrack#1\right\lbrack}
\newcommand{\intof}[1]{\left\rbrack#1\right\rbrack}
\newcommand{\intfo}[1]{\left\lbrack#1\right\lbrack}
\newcommand{\intff}[1]{\left\lbrack#1\right\rbrack}

\DeclareMathOperator{\e}{e}

%====================================================================================================
\begin{document}
%====================================================================================================
\title{
Le chemin du dragon\\
\vspace{0.25cm} \Large \textit{D\'efi math\'ematique adress\' e aux coll\'egiens d\' esireux de prouver leur connaissance des fonctions usuelles}}
\author{par Cyrille \bsc{Praz}}
\date{23.06.2015}
\maketitle
%====================================================================================================
\section{Introduction}
%====================================================================================================
 Le domaine de d\'efinition d'une fonction  r\'eelle   $f$ d'une variable r\'eelle $x$ est le sous-ensemble de $\mathbb{R}$ contenant tous les nombres qui poss\`edent une image par la fonction $f$.
Le but de l'exercice pr\'esent\'e \`a la \cref{exercice} de ce document est  de trouver le domaine de d\'efinition de chacune des 100 fonctions propos\'ees, en moyenne class\'ees par ordre croissant de difficult\'e.
En plus de permettre une r\'evision des fonctions usuelles, cet exercice peut \^etre vu comme un entra\^inement aux \'etudes de signe, aux techniques de factorisation de polyn\^omes et aux m\'ethodes de r\'esolution d'\'equations du premier et du deuxi\`eme degr\'e \`a une inconnue.
La \cref{notations} ci-dessous pr\'esente quelques-unes des notations et conventions utilis\'ees dans ce document ; comme celles-ci sont relativement classiques, la lecture de cette section n'est pas n\'ecessaire dans un premier temps.
Finalement, la \cref{corrige} contient un corrig\'e succinct de l'exercice.
%====================================================================================================
\clearpage
\section{Notations et conventions}\label{notations}
%====================================================================================================
Ci-dessous sont list\'ees quelques notations et conventions utilis\'ees dans ce document :

\begin{enumerate}
\item La fonction arcsinus, not\'ee $\arcsin$, est la r\' eciproque de la  fonction bijective $f:\intff{-\frac{\pi}{2},\frac{\pi}{2}}\rightarrow\intff{-1,1}$ d\'efinie par $f(x)=\sin(x)$.
\item La fonction arccosinus, not\'ee $\arccos$, est la r\' eciproque de la fonction bijective $f:\intff{0,\pi}\rightarrow\intff{-1,1}$ d\'efinie par $f(x)=\cos(x)$.
\item La fonction arctangente, not\'ee $\arctan$, est la r\' eciproque de la fonction bijective $f:\intoo{-\frac{\pi}{2},\frac{\pi}{2}}\rightarrow\mathbb{R}$ d\'efinie par $f(x)=\tan(x)$.
\item Les arguments des fonctions sinus, cosinus et tangente ainsi que les images des fonctions arcsinus, arccosinus et arctangente sont exprim\'es en radians.
\item Pour $n\in\mathbb{N^*}$, la fonction racine 2n-i\`eme, not\'ee $\sqrt[2n]{\cdot}$ ou simplement $\sqrt{\cdot}$ dans le cas $n=1$, est la r\' eciproque de la fonction bijective $f:\mathbb{R_+}\rightarrow\mathbb{R_+}$ d\'efinie par $f(x)=x^{2n}$.
%\item (Faux?) Pour $n\in\mathbb{N}$, la fonction racine (2n+1)-i\`eme, not\'ee $\sqrt[2n+1]{\cdot}$, est la r\' eciproque de la fonction bijective $f:\mathbb{R}\rightarrow\mathbb{R}$ d\'efinie par $f(x)=x^{2n+1}$.
\item Pour  $x\in\mathbb{R^*}$ et $a\in\mathbb{Z^*}$, on \'ecrit $x^{-a}=\frac{1}{x^a}$. Suivant la m\^eme logique, on consid\`ere que si $x\in\mathbb{R^*}$, alors $x^0=1$. De plus, on adopte la convention $0^0=1$.
\item Pour $x\in\mathbb{R_+}$ et $p,q\in\mathbb{N^*}$, on d\'efinit $x^{\frac{p}{q}}=\sqrt[q]{x^p}$. Si de plus $x$ est non nul, on d\'efinit $x^{-\frac{p}{q}}=\frac{1}{\sqrt[q]{x^p}}$.
\item Pour $x\in\mathbb{R_+^*}$ et $y\in\mathbb{R}$, on g\'en\'eralise la d\'efinition de puissance selon la relation $x^y=\e^{y\ln(x)}$. De plus, on g\'en\'eralise la relation $0^z=0$ pour tout $z\in\mathbb{R_+^*}$.
\item Pour $a\in\mathbb{R_+^*}$, la fonction logarithme en base $a$, not\'ee $\log_a$ ou simplement $\log$ dans le cas $a=10$ ou encore $\ln$ dans le cas $a=\e$, est la r\' eciproque de la fonction bijective $f:\mathbb{R}\rightarrow\mathbb{R_+^*}$ d\'efinie par $f(x)=a^x$.
\item Pour $n\in\mathbb{N}$, les notations $\sin^n(x)$ et $\ln^n(x)$ sont respectivement \'equivalentes aux expressions $\left\lbrack\sin(x)\right\rbrack^n$ et $\left\lbrack\ln(x)\right\rbrack^n$.

\end{enumerate}
%====================================================================================================
\clearpage
\section{Exercice}\label{exercice}
Trouver le domaine de d\'efinition des fonctions r\'eelles \`a variable r\'eelle suivantes : 
%====================================================================================================
\begin{multicols}{2}
\begin{enumerate}

% 0
\item $f(x)=1$
\item $f(x)=x$
\item $f(x)=-x$
\item $f(x)=\frac{1}{x}$
\item $f(x)=|x|$
\item $f(x)=2x$
\item $f(x)= \frac{x}{2}$
\item $f(x)=x^2$
\item $f(x)=\sqrt{x}$
\item $f(x)=\frac{1}{\sqrt{x}}$
% 10
\item $f(x)=10^x$
\item $f(x)=\log(x)$
\item $f(x)=\e^x$
\item $f(x)=\ln(x)$
\item $f(x)=\sin(x)$
\item $f(x)=\arcsin(x)$
\item $f(x)=\cos(x)$
\item $f(x)=\arccos(x)$
\item $f(x)=\tan(x)$
\item $f(x)=\arctan(x)$
% 20
\item $f(x)=x^1$
\item $f(x)=x^{\frac{1}{2}}$
\item $f(x)=x^0$
\item $f(x)=x^{-\frac{1}{2}}$
\item $f(x)=x^{-1}$
\item $f(x)=(x+7)^1$
\item $f(x)=(x+7)^{\frac{1}{2}}$
\item $f(x)=(x+7)^0$
\item $f(x)=(x+7)^{-\frac{1}{2}}$
\item $f(x)=(x+7)^{-1}$
% 30
\item $f(x)=x^2+2x+3$
\item $f(x)=\sqrt{3}x^4-x^3+\pi x^2-11x+\e$
\item $f(x)=\frac{1}{x+7}$
\item $f(x)=\frac{1}{2x-\pi}$
\item $f(x)=\frac{1}{x^2+4x-5}$
\item $f(x)=\frac{1}{2x^2-5x+3}$
\item $f(x)=\frac{1}{x^3-x^2-2x}$
\item $f(x)=\frac{\pi x^2 +7x -3}{2x^2-\frac{1}{2}x-\frac{3}{2}}$
\item $f(x)=\frac{6x-1}{x^3-15x^2+75x-125}$
\item $f(x)=\frac{1}{x-1}+\frac{2}{x^2-2}+\frac{3}{x^3-3}$
% 40
\item $f(x)=\frac{1}{\sqrt{-x}}$
\item $f(x)=\sqrt{x^2}$
\item $f(x)=\sqrt{1-2x}$
\item $f(x)=\sqrt{x^2+1}$
\item $f(x)=\sqrt{5-\frac{1}{2}x}+\sqrt{3x+4}$
\item $f(x)=\sqrt{x^2-x-1}$
\item $f(x)=\sqrt{(4x^2-7x)^3}$
\item $f(x)=\sqrt{\frac{1-9x}{4x^4-12x^2+9}}$
\item $f(x)=\sqrt{\frac{1-x^2}{x^3-\frac{1}{4}x}}$
\item $f(x)=\frac{1}{\sqrt{x^2-3}-4}$
% 50
\item $f(x)=\e^{x^2-5x}$
\item $f(x)=\frac{1}{\e^{8x-9}}$
\item $f(x)=\frac{10^x-2}{x-2}$
\item $f(x)=\frac{x-2}{10^x-2}$
\item $f(x)=\sqrt{3^{6-7x}}$
\item $f(x)=\ln\left(x^2-5x\right)$
\item $f(x)=\frac{1}{\ln\left(8x-9\right)}$
\item $f(x)=\frac{\log(x)-2}{x-2}$
\item $f(x)=\frac{x-2}{\log(x)-2}$
\item $f(x)=\sqrt{\log_3(6-7x)}$
% 60
\item $f(x)=\sin(10x-1)$
\item $f(x)=\cos(10x-2)$
\item $f(x)=\tan(10x-3)$
\item $f(x)=\arcsin(10x-4)$
\item $f(x)=\arccos(10x-5)$
\item $f(x)=\arctan(10x-6)$
\item $f(x)=\frac{1}{\sin(10x-7)}$
\item $f(x)=\frac{1}{\cos(10x-8)}$
\item $f(x)=\frac{1}{\tan(10x-9)}$
\item $f(x)=\frac{1}{\arcsin(10x-10)}$
% 70
\item $f(x)=\sqrt{1-\sqrt{x^2}}+\sqrt{1-2x}$
\item $f(x)=\frac{1}{\sqrt{x^2+(\e-\pi)x-\e\pi}}$
\item $f(x)=\sqrt{\frac{x^2-8}{x^3+8}}$
\item $f(x)=\sqrt{1-\sqrt{x^2-1}}$
\item $f(x)=\sqrt{x-|x-2|}$
\item $f(x)=\sqrt{\sin(x)}$
\item $f(x)=\frac{1}{\sqrt{2x-1}-\sqrt{3x-4}}$
\item $f(x)=\sqrt{\sqrt{2-\frac{1}{3}x}-\sqrt{3x+1}}$
\item $f(x)=\sqrt{\frac{2x^2+4x-5}{\sqrt{2}-\sqrt{x+10}}}$
\item $f(x)=\sqrt{\frac{9x-8\sqrt{x}-1}{x-2\sqrt{x}+\frac{3}{4}}}$
% 80
\item $f(x)=\e^{\frac{8t^5-3t^4+7}{t^4+2t^2+1}}$
\item $f(x)=\frac{\sqrt{x+2}}{\e^{7x-1}-\e^{3-8x}}$
\item $f(x)=\sqrt{5-\e^{x^2-5}}$
\item $f(x)=\frac{\pi^x}{(2^x-5)(7^x-11)(\e^x-3)}$
\item $f(x)=\frac{1}{2\cdot3^{6x-3}-7\cdot3^{4x-2}+5\cdot3^{2x-1}}$ %1/x(2x-5)(3x-1)
\item $f(x)=\frac{\ln(2x-3)}{\ln(2x)-3}$
\item $f(x)=\ln\left\lbrack-\frac{\sqrt{6}x^2+(2\sqrt{3}-3\sqrt{2})x-6}{2x^2+3x-5}\right\rbrack$
\item $f(x)=\ln|\log(x)|$
\item $f(x)=\sqrt{\ln\sqrt{3-x^2}}$
\item $f(x)=\frac{\ln\left\lbrack\ln^2(x)+10\ln(x)+25\right\rbrack}{\ln^2(x)-\ln(x)}$
%90
\item $f(x)=\arctan\left\lbrack\sqrt{\log_{2}\left(\frac{2\sin(-\e^\pi)}{7^{-7}}\right)}\,\right\rbrack$
\item $f(x)=\sqrt{x}+\sqrt{\sqrt{3}-\sqrt{5^{x^2+3x}-2}}$
\item $f(x)=\frac{1}{\left|\left|\ln(|x|-1)-2\right|-3\right|}$
\item $f(x)=\frac{\arcsin(x)\arccos(2x)\arctan(3x)}{\sin(4x)\cos(5x)\tan(6x)}$
\item $f(x)=\frac{1}{\ln\left|\sin(x^2+x)\right|}$
\item $f(x)=\frac{\sqrt{-\left|30x^3-x^2-x\right|}}{x^4+\frac{329}{30}x^3+\frac{148}{5}x^2-\frac{41}{30}x-1}$ % x(5x-1)(6x+1)/(x-1/5)(x+5)(x+1/6)(x+6)
\item $f(x)=\sqrt{\frac{14^x-11\cdot7^x+3\cdot2^x-33}{x^5+5x^4+10x^3+10x^2+5x+1}}$
\item $f(x)=\arcsin(x)+\sqrt{\frac{x^2+3|x|+2}{\ln\left|\frac{1}{2}+\cos(3x+2)\right|}}$
\item $f(x)=\frac{\log(5x-4)}{\log(5x-4)}+\frac{\cos(x)-x}{\cos(x)-x}+\frac{x^2-2^x}{x^2-2^x}$
\item $f(x)=\frac{\sqrt{-\ln^4(x)+2\ln^3(x)+\ln^2(x)-2\ln(x)}}{\left|-\sin^4(x)+2\sin^3(x)+\sin^2(x)-2\sin(x)\right|}$

\end{enumerate}
\end{multicols}
%====================================================================================================
\clearpage
\section{Corrig\'e}\label{corrige} 
%====================================================================================================
\begin{multicols}{2}
\begin{enumerate}
% 0
\item $D_f=\mathbb{R}$
\item $D_f=\mathbb{R}$
\item $D_f=\mathbb{R}$
\item $D_f=\mathbb{R^*}$
\item $D_f=\mathbb{R}$
\item $D_f=\mathbb{R}$
\item $D_f=\mathbb{R}$
\item $D_f=\mathbb{R}$
\item $D_f=\mathbb{R_+}$
\item $D_f=\mathbb{R_+^*}$
% 10
\item $D_f=\mathbb{R}$
\item $D_f=\mathbb{R_+^*}$
\item $D_f=\mathbb{R}$
\item $D_f=\mathbb{R_+^*}$
\item $D_f=\mathbb{R}$
\item $D_f=\intff{-1,1}$
\item $D_f=\mathbb{R}$
\item $D_f=\intff{-1,1}$
\item $D_f=\mathbb{R}\minuss{\frac{\pi}{2}+k\pi : k\in\mathbb{Z}}$
\item $D_f=\mathbb{R}$
% 20
\item $D_f=\mathbb{R}$
\item $D_f=\mathbb{R_+}$
\item $D_f=\mathbb{R}$
\item $D_f=\mathbb{R_+^*}$
\item $D_f=\mathbb{R^*}$
\item $D_f=\mathbb{R}$
\item $D_f=\intfo{-7,+\infty}$
\item $D_f=\mathbb{R}$
\item $D_f=\intoo{-7,+\infty}$
\item $D_f=\mathbb{R}\minuss{-7}$
% 30
\item $D_f=\mathbb{R}$
\item $D_f=\mathbb{R}$
\item $D_f=\mathbb{R}\minuss{-7}$
\item $D_f=\mathbb{R}\minuss{\frac{\pi}{2}}$
\item $D_f=\mathbb{R}\minuss{-5,1}$
\item $D_f=\mathbb{R}\minuss{1,\frac{3}{2}}$
\item $D_f=\mathbb{R}\minuss{-1,0,2}$
\item $D_f=\mathbb{R}\minuss{-\frac{3}{4},1}$
\item $D_f=\mathbb{R}\minuss{5}$
\item $D_f=\mathbb{R}\minuss{1,\pm\sqrt{2},\sqrt[3]{3}}$
% 40
\item $D_f=\mathbb{R_-^*}$
\item $D_f=\mathbb{R}$
\item $D_f=\intof{-\infty,\frac{1}{2}}$
\item $D_f=\mathbb{R}$
\item $D_f=\intff{-\frac{4}{3},10}$
\item $D_f=\mathbb{R}\minusioo{\frac{1-\sqrt{5}}{2},\frac{1+\sqrt{5}}{2}}$
\item $D_f=\mathbb{R}\minusioo{0,\frac{7}{4}}$
\item $D_f=\intof{-\infty,\frac{1}{9}}\minuss{-\sqrt{\frac{3}{2}}}$
\item $D_f=\intof{-\infty,-1}\cup\intoo{-\frac{1}{2},0}\cup\intof{\frac{1}{2},1}$
\item $D_f=\left(\mathbb{R}\minusioo{-\sqrt{3},\sqrt{3}}\right)\minuss{\pm\sqrt{19}}$
% 50 
\item $D_f=\mathbb{R}$
\item $D_f=\mathbb{R}$
\item $D_f=\mathbb{R}\minuss{2}$
\item $D_f=\mathbb{R}\minuss{\log(2)}$
\item $D_f=\mathbb{R}$
\item $D_f=\mathbb{R}\minusiff{0,5}$
\item $D_f=\intoo{\frac{9}{8},+\infty}\minuss{\frac{5}{4}}$
\item $D_f=\mathbb{R_+^*}\minuss{2}$
\item $D_f=\mathbb{R_+^*}\minuss{100}$
\item $D_f=\intof{-\infty,\frac{5}{7}}$
 % 60
\item $D_f=\mathbb{R}$
\item $D_f=\mathbb{R}$
\item $D_f=\mathbb{R}\minuss{\frac{1}{10}\left(3+\frac{\pi}{2}+k\pi\right) : k\in\mathbb{Z}}$
\item $D_f=\intff{\frac{3}{10},\frac{1}{2}}$
\item $D_f=\intff{\frac{2}{5},\frac{3}{5}}$
\item $D_f=\mathbb{R}$
\item $D_f=\mathbb{R}\minuss{\frac{1}{10}\left(7+k\pi\right) : k\in\mathbb{Z}}$
\item $D_f=\mathbb{R}\minuss{\frac{1}{10}\left(8+\frac{\pi}{2}+k\pi\right) : k\in\mathbb{Z}}$
\item $D_f=\mathbb{R}\minuss{\frac{1}{10}\left(9+k\frac{\pi}{2}\right) : k\in\mathbb{Z}}$
\item $D_f=\intff{\frac{9}{10},\frac{11}{10}}\minuss{1}$
% 70
\item $D_f=\intff{-1,\frac{1}{2}}$
\item $D_f=\mathbb{R}\minusiff{-\e,\pi}$
\item $D_f=\intfo{-2\sqrt{2},-2}\cup\intfo{2\sqrt{2},+\infty}$
\item $D_f=\intff{-\sqrt{2},-1}\cup\intff{1,\sqrt{2}}$
\item $D_f=\intfo{1,+\infty}$
\item $D_f=\set{a+2k\pi: a\in\intff{0,\pi}, k\in\mathbb{Z}}$
\item $D_f=\intfo{\frac{4}{3},+\infty}\minuss{3}$
\item $D_f=\intff{-\frac{1}{3},\frac{3}{10}}$
\item $D_f=\intfo{-10,-8}\cup\intff{\frac{-2-\sqrt{14}}{2},\frac{-2+\sqrt{14}}{2}}$
\item $D_f=\intof{\frac{1}{4},1}\cup\intoo{\frac{9}{4},+\infty}$
% 80
\item $D_f=\mathbb{R}$
\item $D_f=\intfo{-2,+\infty}\minuss{\frac{4}{15}}$
\item $D_f=\intff{-\sqrt{5+\ln(5)},\sqrt{5+\ln(5)}}$
\item $D_f=\mathbb{R}\minuss{\log_2(5),\log_7(11),\ln(3)}$
\item $D_f=\mathbb{R}\minuss{\frac{1}{2},\frac{\log_3\left(\frac{5}{2}\right)+1}{2}}$
\item $D_f=\intoo{\frac{3}{2},+\infty}\minuss{\frac{\e^3}{2}}$
\item $D_f=\intoo{-\frac{5}{2},-\sqrt{2}}\cup\intoo{1,\sqrt{3}}$
\item $D_f=\mathbb{R_+^*}\minuss{1}$
\item $D_f=\intff{-\sqrt{2},\sqrt{2}}$
\item $D_f=\mathbb{R_+^*}\minuss{\frac{1}{\e^5},1,\e}$
 % 90
\item $D_f=\mathbb{R}$
\item $D_f=\intff{\frac{\sqrt{9+4\log_5(2)}-3}{2},\frac{\sqrt{15}-3}{2}}$
\item $D_f=\left(\mathbb{R}\minusiff{-1,1}\right)\minuss{\pm(\e^{-1}+1),\pm(\e^5+1)}$
\item $D_f=\intff{-\frac{1}{2},\frac{1}{2}}\minuss{0,\pm\frac{\pi}{10},\pm\frac{\pi}{12}}$
\item $D_f=\mathbb{R}\minuss{\frac{-1\pm\sqrt{1+2n\pi}}{2}: n\in\mathbb{N}}$
\item $D_f=\set{0}$
\item $D_f=\mathbb{R}\minusifo{-1,\log_2(11)}$
\item $D_f=\intoo{-\frac{\pi+8}{12},\frac{\pi-8}{12}}$
\item $D_f=\intoo{\frac{4}{5},+\infty}\minuss{2,4}$
\item $D_f=\intff{\frac{1}{\e},1}\cup\intff{\e,\e^2}\minuss{\pi,\frac{3\pi}{2},2\pi}$

\end{enumerate}
\end{multicols}
%====================================================================================================
\end{document}
%====================================================================================================
